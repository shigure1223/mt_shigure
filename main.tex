\documentclass[a4paper, 14pt]{ltjsreport}

\usepackage{luatexja-fontspec}
\usepackage[ipa]{luatexja-preset}

% 画像系
\usepackage{graphicx}
\graphicspath{ {./figures/} }
\usepackage{caption}
\usepackage{subcaption}
% 現時点で使っていない
% \usepackage[percent]{overpic}

\usepackage{ifthen}

% 数式系
\usepackage{amsmath}
\usepackage{bm}

% 参考文献リストの設定
% 参考文献はreferences.bibに記述する
\usepackage[backend=biber,style=phys]{biblatex}
\addbibresource{references.bib}

% ハイパーリンクの設定
\ifthenelse{
  \equal{\directlua{tex.sprint(os.getenv("DRAFT"))}}{1}
}{
  \usepackage[%
  luatex,%
  pdfencoding=auto,%
  colorlinks=true,%
  pdfauthor={杉浦 航},%
  pdftitle={温度感受性イオンチャネルタンパク質の物性と分子機能},%
  pdfsubject={温度感受性イオンチャネルタンパク質の物性と分子機能},%
  pdfkeywords={TRP, 熱伝導度, 分子動力学シミュレーション},%
  ]{hyperref}
}{
  \usepackage[%
  luatex,%
  pdfencoding=auto,%
  pdfauthor={杉浦 航},%
  pdftitle={温度感受性イオンチャネルタンパク質の物性と分子機能},%
  pdfsubject={温度感受性イオンチャネルタンパク質の物性と分子機能},%
  pdfkeywords={TRP, 熱伝導度, 分子動力学シミュレーション},%
  ]{hyperref}
}

% いい感じに定義
\newcommand{\openFortyTwo}{開構造モデル}
\newcommand{\closeFortyTwo}{閉構造モデル}
\newcommand{\molNamePOPC}{1-palmitoyl-2-oleoyl-sn-glycero-3-phosphocholine}
\newcommand{\molNamePOPE}{1-palmitoyl-2-oleoyl-phosphatidylethanolamine}
\newcommand{\molNameCHOL}{Cholesterol}

\title{温度感受性イオンチャネルタンパク質の物性と分子機能}
\author{名古屋大学理学研究科理学専攻\\杉浦航}
\date{\today}

\begin{document}

% 表紙
\maketitle

\chapter*{概要}
タンパク質のアロステリー現象は、リガンド結合や外部刺激によって生じる構造変化が活性部位に影響を及ぼす現象であり、そのメカニズム解明は生命現象の理解と創薬研究において重要な課題である。
本研究では、Gタンパク質共役受容体(GPCR)の一種である$\beta$2アドレナリン受容体(β2AR)を対象とし、アロステリー機構を明らかにするための動的ネットワーク解析を実施した。

グラフ理論を基盤とする従来の解析が抱える、動的性質の反映不足や最短経路以外の経路の無視という問題を部分的に克服するため、分子動力学(MD)シミュレーションを用いて得られたトラジェクトリーから、時間依存的な残基間相互作用を反映したネットワークを構築した。
さらに、Louvain法によるコミュニティ検出を適用することで、タンパク質の協調的な動きによるシグナル伝達機構を多角的に解析した。

その結果、active状態では新たに形成されたコミュニティが観察され、これがアロステリックシグナル伝達において重要な役割を果たすことを示唆した。
また、保存された水分子が構造安定性およびシグナル伝達に果たす役割を確認した。

本研究は、タンパク質の動的性質を考慮したネットワーク解析手法が、アロステリー機構の理解に貢献する可能性を示すものである。

% 目次
\tableofcontents
\clearpage

\chapter{序論}
\section{はじめに}

タンパク質はアミノ酸が多数繋がって構成されている高分子化合物であり、タンパク質全体が分子機械として働く。しかしこのアミノ酸単位やアミノ酸間の相互作用という”部分”としての局所的挙動とドメイン単位やタンパク質という”全体”としての大域的挙動には時空間の大きな隔たりがある。
それが顕著に表れている具体的な話でいうと、タンパク質のアロステリー現象が挙げられる。
アロステリーとは、分子の一部が離れた他の部分に影響を及ぼす現象であり、タンパク質の機能を制御する重要な特性である。具体的には、リガンド結合など外部刺激の効果によってシグナルが残基間相互作用を介して活性部位に伝わり、タンパク質を活性化させる。
この過程は、サブピコ秒からミリ秒の時間スケールにわたるダイナミクスと、サブ\text{\AA}~数十\text{\AA}の空間スケールの相互作用が連動して行われることが興味深い点であり大きな謎となっている。
%参考文献:アロステリー、時間スケールと空間スケール
%2005「シグナル伝達のアロステリック機構」
%https://www.science.org/doi/abs/10.1126/science.1108595
アロステリーのメカニズムを解明するの一つの方法として、\textbf{グラフ理論}を用いたアプローチが注目されている。
グラフ理論は、分子内の残基間の相互作用をネットワークとして表現することができ、複雑な動的挙動を解析するための協力なツールとして広く利用されてきた。
アロステリー機構に関しては、中心性の高い残基やネットワーク上における最短経路上によく現れる残基を解析することで、タンパク質のシグナル伝達の効率的な経路の同定に有用な知見を与えてきた。
%参考文献:グラフ理論
%2011「タンパク質構造の残基ネットワークの解析と可視化」
%https://www.cell.com/trends/biochemical-sciences/abstract/S0968-0004(11)00013-2?_returnURL=https%3A%2F%2Flinkinghub.elsevier.com%2Fretrieve%2Fpii%2FS0968000411000132%3Fshowall%3Dtrue
%2011「RING: タンパク質構造における相互作用残基、進化情報、エネルギーのネットワーク化」
%https://academic.oup.com/bioinformatics/article/27/14/2003/193801?login=false
%2012「生物学的ネットワークとタンパク質構造のトポロジー解析とインタラクティブな可視化」
%https://www.nature.com/articles/nprot.2012.004

%従来のグラフ理論の説明
しかし、従来のグラフ理論に基づく解析方法では、タンパク質の動的変化を十分に反映できなかったり、シグナルが特定の最短経路を通るといういわば決定論すぎるという問題点がある。

そこで本研究では、アロステリーの解析対象として\textbf{β2アドレナリン受容体(β2AR)}を選定し、inactive状態とactive状態におけるトラジェクトリー解析を基盤とした「動的ネットワーク」の構築した。さらに、部分的に確率的な性質を反映したlouvain法による「コミュニティ検出」を通じて、アロステリック効果に関連する分子内シグナル伝達の機構の解明に努めた。
その結果、active状態で新たに形成されたコミュニティが確認され、密度を用いた定量的な解析によってその役割が明らかになった。
また、\textbf{β2アドレナリン受容体(β2AR)}特有の保存された水の重要性も示唆される結果を得ることができた。


本論文の構成は以下の通りである。
はじめに、本研究の対象タンパク質であるβ2ARについて\ref{sec:b2ar}節で記述する。
次に、グラフ理論を本研究にどのように適用したかを\ref{sec:graph theory}節で説明する。
\ref{chap:methods}章以降は、シミュレーションと解析の方法とその結果、考察をまとめる。

\newpage

\section{β2アドレナリン受容体(β2AR)}\label{sec:b2ar}
β2アドレナリン受容体(β2AR)は、Gタンパク質共役受容体(GPCR)の一種であり、細胞膜に存在する重要な受容体である。β2ARは7回膜貫通構造を持ち、リガンド(アドレナリンやノルアドレナリンなど)と結合することで構造的変化を引き起こし、細胞内のシグナル伝達を調節する。
β2ARのシグナル伝達機構は、細胞外の刺激(アドレナリンやノルアドレナリン)を細胞内のシグナルに変換するプロセスである。
アドレナリンやノルアドレナリンといったリガンド結合により、β2ARは構造的な変化を起こし、Gタンパク質と結びつき、下流のシグナル伝達経路を活性化させる。この過程を通じて、細胞内の様々なシグナルが伝達され、最終的にエネルギー供給や筋肉緊張の調節、呼吸制御、循環の改善など、幅広い生理的応答が引き起こされる。これらの機能は、心血管疾患や喘息などの治療薬開発において重要なターゲットとなっており、β2ARは医薬品開発の重要な候補分子でもある。
%詳細の構造と、リガンド・gタンパク質結合部位を示す

\subsubsection*{β2ARの初期構造および特有の残基} 
β2ARの構造は、X線結晶解析やNMR解析によって明らかにされており、その構造的特徴には、アロステリックシグナル伝達に関与する特有の残基が存在する。これらの残基の相互作用を理解することが、アロステリーのメカニズム解明に繋がると考えられる。
%モチーフと水の話

\section{グラフ理論}\label{sec:graph theory}
\documentclass{article}
\usepackage{amsmath}

\begin{document}

\section*{グラフ理論の基礎}

グラフ理論は、ローカルな情報とグローバルな情報の両方を活用して、シグナル伝達やアロステリック制御に重要な役割を果たすタンパク質の残基(=機能的残基)を同定する手法として導入されている。このアプローチは、タンパク質の構造やダイナミクスをネットワークの観点から捉えることで、直感的に理解しやすい特徴を提供している。
グラフ理論においては、タンパク質内のアミノ酸残基の相互作用を、ノードとエッジを用いたアミノ酸ネットワークとして表現することができる。

\begin{itemize}
    \item \textbf{ノード}:アミノ酸残基(原子やタンパク質全体も可能)
    \item \textbf{エッジ}:ノード同士の相互作用
\end{itemize}

この方法を用いて、タンパク質内の相互作用の構造を視覚化し、機能的に重要な残基を特定することができる。特に、機能的残基の同定には以下の二つの手法が一般的である。

\subsection*{静的ネットワークと最短距離解析}

1. \textbf{静的ネットワーク}:構造データから得られた時間依存しない変数を重みとしてネットワークを構築する。このネットワークはタンパク質の静的な相互作用を反映し、構造に基づく機能的残基を同定するために用いられる。
2. \textbf{最短距離解析}:最短経路上でよく現れる残基を、機能的に重要な残基として同定する手法である。これにより、タンパク質内でシグナルを伝達する上で重要な経路を特定することができる。

しかし、これらの手法にはいくつかの限界がある。

\subsection*{問題点}

1. \textbf{タンパク質の動的性質を反映できない}  
   静的ネットワークや最短経路解析は、タンパク質の動的変化を十分に反映できない。アロステリック変化(例えば、リガンド結合による構造変化や、Gタンパク質との相互作用によるコンフォメーション変化)のメカニズムを捉えることが難しい。
2. \textbf{最短経路以外の潜在的経路の無視}  
   最短経路解析では、タンパク質内でシグナルが伝達される経路の中で、エネルギー的に有利な経路や微細な構造変化を通じて伝わる経路を反映することができんじ。このため、最短経路以外の潜在的な経路や全体構造を反映できないという問題がある。
3. \textbf{タンパク質全体としての協調的な動きの捉えにくさ}  
   アロステリーはタンパク質の複数の部分が連携して機能する現象であるため、タンパク質全体としての協調的な動きや、多様な経路による相互作用を捉えることが難しい。

これらの問題を解決するためには、より動的で柔軟な手法を用いる必要がある。

\chapter{材料と方法}\label{chap:methods}
% 方法と材料を述べる
% 今回の研究では、次のテクニックを使った。
% - RMSD
% - RMSF
% - Thermal conductivity by curp
%   - 熱流の理論式
%   - 熱伝導度の理論式
% - MD simulation
%   - AMBER
%   - PME
%   - NPgT
%   - NPT
%   - minimization
%   - equilibration
%   - production
%   - NVE

% アウトライン
% - 熱伝導度の理論を説明する
%   - 熱伝導度の計算方法を説明する
% - 熱伝導度を計算するための準備としての、分子動力学シミュレーションの方法を説明する
%   - エネルギー最小化
%   - 熱平衡化
%   - sampling
%   - NVE

\section{熱伝導度の解析}

\section{分子動力学シミュレーション}

熱伝導度を計算するためには、NVEアンサンブルでの分子動力学シミュレーションを行う必要がある。
そのため本研究では、モデリング、構造最適化、熱平衡化、サンプリング、解析の5つのステップを踏んで熱伝導度の計算を行った。

\subsection{モデリング}
開構造として7MIO、% FLOW_ID = 13, structure_id = 11
閉構造として7MIN  % FLOW_ID = 11, structure_id = 9 
を用いた。\autocite{noauthor_7mio_nodate, noauthor_7min_nodate, nadezhdinStructuralMechanismHeatinduced2021}
それぞれの構造について、CHARMM-GUIのInput generator\autocite{jo_charmmgui_2008, lee_charmm-gui_2016}を用いて、リン脂質二重膜にタンパク質を挿入して、シミュレーションボックスを作成した。
なお、今後開構造の7MIOを「\openFortyTwo」、閉構造の7MINを「\closeFortyTwo」と呼ぶことにする。

CHARMM-GUIを用いて、次のように系を作成した。
まず、\openFortyTwo は残基番号77から112までの残基はモデリングされていなかったため、残基番号113以降の構造情報のみを用いた。
次に、PDBデータに含まれていたタンパク質以外の分子についてはあらかじめすべて除去した。
また、両方のモデルについて、タンパク質分子に対して pH 7.0 でプロトン化した。
続けて、作成したモデルをリン脂質二重層に挿入した。
% メモ: 脂質の名前に自信がない。
% メモ: newcommandで定義しているので、後で変更するときは定義場所だけ変更すればよい。
% POPC
%  - 1-palmitoyl-2-oleoyl-sn-glycero-3-phosphocholine, https://en.wikipedia.org/wiki/POPC
% POPE
%  - 1-palmitoyl-2-oleoyl-sn-glycero-3-phosphoethanolamine, https://avantilipids.com/product/850757
%  - palmitoyloleoyl-phosphatidylethanolamine, https://www.ncbi.nlm.nih.gov/pmc/articles/PMC1305115/
%  - 1-Palmitoyl-2-oleoylphosphatidylethanolamine, https://pubchem.ncbi.nlm.nih.gov/compound/1-Palmitoyl-2-oleoylphosphatidylethanolamine
%  - 1-palmitoyl-2-oleoyl-phosphatidylethanolamine, https://pubs.acs.org/doi/10.1021/bi060937y
脂質二重層の構成分子は、\molNamePOPC (POPC)、\molNamePOPE (POPE)、\molNameCHOL (CHOL)を使い、これらの分子をPOPC:POPE:CHOL=2:1:1の比率で混合した。
最後に、水分子とNaClイオンを加えて系の電荷を中和した。NaClイオンは0.15 Mの濃度になるように追加した。
シミュレーションパッケージAMBER 22\autocite{case_amber_2023}を使ったため、ポテンシャル関数としてAMBER\autocite{pearlman_amber_1995}を利用した。
ポテンシャル関数の各パラメータは、
タンパク質の原子にはff19SB\autocite{tian_ff19sb_2020}、
脂質の原子にはLipid21\autocite{dickson_lipid21_2022}、
水分子にはOPC水モデル\autocite{izadi_building_2014}を用いた。
系には3次元の周期境界条件を設定した。
系を作成した後の系の大きさを表\ref{tab:system_size}に示す。

\begin{table}[!ht]
  \centering
  \caption{シミュレーションに用いた系の大きさ}
  \begin{tabular}{lll}
    \hline
    モデル名        & 原子数  & ボックスの大きさ(x, y, z) \\
    \hline 
    \openFortyTwo  & 1183281 & 250.19627, 250.19627, 161.529 \\ 
    \closeFortyTwo & 1206767 & 250.14921, 250.14921, 164.691 \\ 
  \end{tabular}
  \label{tab:system_size}
\end{table}

\subsection{シミュレーション}
作成したモデルは、以下の手続きによりシミュレーションを行った。シミュレーションには、Particle Mesh Ewald法を用いた。
% 粒子として取り扱う距離を cut = 9 [A]とした。

はじめに、構造最適化を行った。構造最適化は3つのステップに分けて行った。
最初のステップでは、水以外の原子に対してrestraintをかけて、水の構造、並びを最適化した。
次のステップでは、タンパク質の原子に対してrestraintをかけて、水と脂質の構造を最適化した。
最後のステップでは、すべてのrestraintを外して、全体の構造を最適化した。
各ステップにおいて、最初の2500ステップで再急降下法によるエネルギー最小化を行い、次の2500ステップで共役勾配法によるエネルギー最小化を行った。

次に、熱平衡化を行った。
熱平衡化は大きく分けて2つのステップに分けて行った。
% 1. T = 10K で開始。NVT. 初期速度はマクスウェル分布で与える。1.mdinと2.mdinを使って、dt = 1 fsで125000 + 125000 = 250 psのシミュレーションの中で10 K -> 300Kまで上げる。
% 2. NPgT. 125000 steps, dt = 1 fs. 3.mdin
% 3. NPgT. 250000 steps, dt = 2 fs. 4.mdin
% 4. 
初めに、NVTアンサンブルで系の温度を上げた。初期速度は温度$T = 10 \rm{[K]}$のマクスウェル分布に従って与え、250 $\rm{ps}$で温度を$T = 315.15 \rm{[K]}$まで上昇させた。
次に、NP\gamma Tアンサンブルで平衡化を行った。
NP\gamma Tアンサンブルでは、温度、膜の表面張力、系の総数が一定になるようにシミュレーションを行う。% TODO: NPgTの説明を書く
詳細なシミュレーション条件は、表\ref{tab:simulation_condition}に示す。

% TODO: restraintについて記述する。
\begin{table}[!ht]
  \centering
  \caption{シミュレーション条件}
  \begin{tabular}{lllll}
    \hline
    ステップ & \Delta t [fs] & 時間 [ps] & アンサンブル & 温度 [T] \\
    \hline
    1       & 1              & 125       & NVT         & 10 $\rightarrow$ 100 \\
    2       & 1              & 125       & NVT         & 100 $\rightarrow$ 315.15 \\
    3       & 1              & 125       & NP\gamma T  & 315.15 \\
    4       & 2              & 500       & NP\gamma T  & 315.15 \\
    5       & 2              & 500       & NP\gamma T  & 315.15 \\
    6       & 2              & 500       & NP\gamma T  & 315.15 \\
  \end{tabular}
  \label{tab:simulation_condition}
\end{table}

続いて、NP\gamma Tシミュレーションによるサンプリングを行った。
熱平衡化の最後のステップと同じ、ただしrestraintを外した状態で10 nsのシミュレーションを実行した。
最後の5 nsの中で、各500 psごとに座標と速度を保存し、次のNVEのシミュレーションの初期状態として用いた。

最後にNVEシミュレーションによるサンプリングを行った。
初期状態として、NP\gamma Tシミュレーションによるサンプリングで得られた座標と速度を用いた。
保存した座標と速度のうち、5つを使って\Delta t = 1 fs、1 nsのシミュレーションを実行した。
シミュレーション中、各10 fsごとにトラジェクトリを保存した。

\section{熱伝導度の解析}

NVEシミュレーションによって得られたトラジェクトリを用いて、熱伝導度を計算した。
熱伝導度の計算には、CURP\autocite{ishikura_energy_2015,ota_energy_2019,yamatoComputationalStudyThermal2022,wangSiteselectiveHeatCurrent2023} を用いた。
なお、今回の研究を遂行するにあたり、熱流の計算速度を改善する必要があったため、改良を施した。改良方法については、\ref{sec:curp}節で述べる。



\subsection{CURPの改良}\label{sec:curp}

TBW


\chapter{結果と考察}
\section{\openFortyTwo と\closeFortyTwo のネットワークの比較}

まず、$\Delta \lambda_{\rm{AB}}$が大きいペアと小さいペアを調べるために、$\Delta \lambda_{\rm{AB}}$ の $\mu \pm 3 \sigma$ よりも外側のペアを抽出した。
図\ref{fig:network_groups_tc_Delta}にドメイン間の$\Delta \lambda_{\rm{AB}}$に着目した図を示す。また、図

\begin{figure}
  \centering
  \includegraphics[width=\textwidth]{network_groups_tc_Delta.png}
  \caption{\openFortyTwo と\closeFortyTwo の $\Delta \lambda_{\rm{AB}}$をネットワーク表示した。
            四角形は主鎖Bの残基。ドメインごとに色を分け、N末端側から順にARD、N-linker、pre-S1、S1-S4、S4-S5-linker、S5-PH-S6、TRP-helix、CTDと分けた。
            \autocite{pumroy_structural_2020}}
  \label{fig:network_groups_tc_Delta}
\end{figure}

$\Delta \lambda_{\rm{AB}}$が大きいペアは、

\begin{figure}
  \centering
  \includegraphics[width=\textwidth]{network_groups_tc_Delta_focus_inside.png}
  \caption{$\Delta \lambda_{\rm{AB}}$が大きいペアのうち、ドメイン内のネットワーク}
  \label{fig:network_groups_tc_Delta_large}
\end{figure}

\chapter{まとめ}
本研究では、$\beta$2ARのinactive状態およびactive状態におけるネットワーク構造を解析し、Louvain法を用いて検出されたコミュニティの特性を比較した。
まず検出されたコミュニティを比較すると、Gタンパク質結合部位が再編成され、新規のコミュニティが形成されていることが明らかになった。
続いてそれぞれのネットワークの全体エッジ密度、コミュニティ内エッジ密度、コミュニティ間エッジ密度を比較した。
全体エッジ密度に関しては、inactive状態では0.3083、active状態では0.8579であった。active状態のネットワークの方がより高い相互作用密度を示し、構造的および機能的に密接な結びつきが形成されていることが確認された。
全てのコミュニティでコミュニティ内エッジ密度が向上し、active状態ではすべてのコミュニティ内エッジ密度が1.0付近という高い値を示した。
コミュニティ間エッジ密度の解析では、ほとんどのコミュニティ間の相互作用が強化されており、特に新しく生成されたnewコミュニティに関連するコミュニティペアは高い値を示した。
これらの結果は、active状態への遷移に伴う分子全体のネットワーク再編成が、分子内情報伝達の効率化を支える重要なメカニズムであることを示している。
特に、newコミュニティが形成されたことで、リガンド結合部位や活性部位であるGタンパク質結合部位間の情報伝達を促進する導管として働いている可能性があることが示唆された。

最後にノード削除が全体エッジ密度とコミュニティ内エッジ密度に与える影響の解析では、リガンド結合部位やnewコミュニティに属しているモチーフに関するノードが全体エッジ密度に、リガンドやgタンパク質結合部位、保存された結晶水がコミュニティ内エッジ密度に与える影響が大きかった。
前者はネットワーク全体の「骨格」としてアロステリックな影響を広げる重要な起点となっていることが、後者は局所的な「柔軟性」を提供しGタンパク質の結合やシグナル伝達の効率化を高めていることが示唆された。

\section{今後の展望}
本研究ではノード間の距離を重みとした構造ネットワークを用いた。
しかし活性化によってダイナミクスや相互作用のみが変化したノードに関しては、構造ネットワークではその変化を捉えることが困難である。
そのため、構造のみならず、ダイナミクスや相互作用も反映した変数である熱伝導度\cite{yamato2022computational}を重みとした物理的な熱ネットワークの構築により、$\beta$2ARのアロステリー機構の解明をより詳細に理解することが期待できる。
%参考文献:熱伝導度
%https://pubs.acs.org/doi/10.1021/acs.jpcb.2c00958
また、Louvain法によるコミュニティ検出では、特定の時間スケールでの1つのコミュニティ分割しか検出しておらず、異なる時間スケールでの過渡現象を観察できない。
そのため、マルコフ安定性\cite{amor2014uncovering}のようなネットワーク内のさまざまなスケールに存在する多層コミュニティ構造を識別できる動力学ベースのマルチスケール方の導入が必要である。
%参考文献:マルコフ安定性
%https://pubs.rsc.org/en/content/articlehtml/2014/mb/c4mb00088a?casa_token=HGfB0iRp9w0AAAAA:3-CTB2Oe4qIicEKrQcC2P6ekaNArGHCwe3FlWDLugZpZvLBt1sOqi5ziQJed1dOzFS6kOYXHzWU-jQ

\chapter*{謝辞}
\addcontentsline{toc}{chapter}{謝辞}
学部3年生の後期セミナーから、学部4年生の卒業研究、博士前期課程2年間の研究活動にいたるまで、倭剛久先生には多大なるご指導、ご助言をいただきました。
深く感謝申し上げます。
また、旧TB研、現B研の皆様には、研究を進めるうえで多くの助言や交流をいただき、大変お世話になりました。
特に、王婷婷氏には、分子動力学シミュレーションや第一原理計算の基礎、研究に対する姿勢などを英語で教えていただき、
たくさんのスキルを身につけさせていただきました。
また、同期の在田陽一氏、下岡渉氏には、普段の研究生活において議論や雑談を通じて多くの刺激をもらうとともに、研究室生活を楽しく過ごすことができました。
岡本祐幸先生、木村明洋先生にも、講義やセミナー、研究室での議論を通じて、多くのご指導をいただきました。
最後に、学生生活の前提となった、安定した生活環境を提供してくださった両親に深く感謝いたします。


% 参考文献
\printbibliography[title=参考文献]
\addcontentsline{toc}{chapter}{参考文献}

\end{document}
