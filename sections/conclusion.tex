本研究では、$\beta$2ARのinactive状態およびactive状態におけるネットワーク構造を解析し、Louvain法を用いて検出されたコミュニティの特性を比較した。
まず検出されたコミュニティを比較すると、gタンパク質結合部位が再編成され、新規のコミュニティが形成されていることが明らかになった。
続いてそれぞれのネットワークの全体エッジ密度、コミュニティ内エッジ密度、コミュニティ間エッジ密度を比較した。
全体エッジ密度に関しては、inactive状態では0.3083、active状態では0.8579であった。active状態のネットワークの方がより高い相互作用密度を示し、構造的および機能的に密接な結びつきが形成されていることが確認された。
全てのコミュニティでコミュニティ内エッジ密度が向上し、特にgproteinコミュニティが顕著に増加した。
コミュニティ間エッジ密度の解析では、リガンド結合部位や活性部位に関連するコミュニティ間の相互作用が強化されており、活性状態における重要な相互作用ネットワークが形成されていることが分かった。
最後にノード削除が全体エッジ密度とコミュニティ内エッジ密度に与える影響の解析では、リガンド結合部位やnewコミュニティに属しているモチーフに関するノードが全体エッジ密度に、リガンドやgタンパク質結合部位、保存された結晶水がコミュニティ内エッジ密度に与える影響が大きかった。
前者はネットワーク全体の「骨格」としてアロステリックな影響を広げる重要な起点となっていることが、後者は局所的な「柔軟性」を提供しGタンパク質の結合やシグナル伝達の効率化を高めていることが示唆された。

\section{今後の展望}
本研究ではノード間の距離を重みとした構造ネットワークを用いた。
しかし活性化によって相互作用のみが変化したノードに関しては、構造ネットワークではその変化を捉えることが困難である。
そのため、構造のみならず、ダイナミクスや相互作用も反映した変数である熱伝導度を重みとした物理的な熱ネットワークの構築により、$\beta$2ARのアロステリー機構の解明をより詳細に理解することが期待できる。