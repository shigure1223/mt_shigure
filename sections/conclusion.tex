本研究では、TRPV3チャネルの残基間相互作用と機能発現機構を理解するために熱流解析を用いたアプローチを採用した。
TRPV3チャネルは温度変化に応答する性質を持ち、特定の温度で活性化する。
このチャネルの構造と機能の関係を明らかにすることは、温度感知メカニズムの理解に寄与するだけでなく、
TRPV3チャネルが関与する病態の理解と治療法の開発につながる可能性がある。
閉じた状態と開いた状態のTRPV3チャネルに対して熱流解析を行い、
Energy Exchange Networkモデルの考えに基づいて、熱の流れを利用してタンパク質内の情報伝達と相互作用を定量的に理解した。
解析の結果、開構造モデルと閉構造モデルにおけるドメインの相互作用が大きく異なることが明らかになり、
熱流ネットワークにおいてドメインが大きく二つに分割されていることが示された。
また、機能に重要な残基な役割を果たすと考えられている残基を調べたところ、
一つの残基ペアにおいて $\Delta \lambda_{\rm{AB}}$ が極端に小さくなることがわかり、
この残基ペアが機能に対してどのような影響を与えるのかを調べることが今後の課題となった。

\section{今後の課題}

本研究では、TRPV3チャネルの\openFortyTwo と\closeFortyTwo に対して熱流解析を行ったが、
TRPV3は開状態と閉状態の二つの状態の間の中間的な状態をとることが知られている。
開状態と閉状態とでは、立体構造に大きな変化が伴う一方、中間的な状態と開状態とでは立体構造に大きな変化がない。
EENから発展した手法は、立体構造の変化が生じないにも関わらず機能に変化が生じるような状態に対して特に効力を発揮する。
そのため、今回の手法を中間的な状態にも適用することで、TRPV3チャネルの機能発現機構をより詳細に理解することができると考えられる。
