学部3年生の後期セミナーから、学部4年生の卒業研究、博士前期課程2年間の研究活動にいたるまで、倭剛久先生には多大なるご指導、ご助言をいただきました。
深く感謝申し上げます。
また、旧TB研、現B研の皆様には、研究を進めるうえで多くの助言や交流をいただき、大変お世話になりました。
特に、王婷婷氏には、分子動力学シミュレーションや第一原理計算の基礎、研究に対する姿勢などを英語で教えていただき、
たくさんのスキルを身につけさせていただきました。
また、同期の在田陽一氏、下岡渉氏には、普段の研究生活において議論や雑談を通じて多くの刺激をもらうとともに、研究室生活を楽しく過ごすことができました。
岡本祐幸先生、木村明洋先生にも、講義やセミナー、研究室での議論を通じて、多くのご指導をいただきました。
最後に、学生生活の前提となった、安定した生活環境と精神的な支えを提供してくださった両親に深く感謝いたします。
