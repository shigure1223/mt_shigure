% 研究の背景、動機、目的を述べる。
% 論文全体の見通しを示す。何章で何を述べるかを明確にする。

% 研究の背景
% - 生体と温度の関係
% - TRPチャネル全体について
%   - TRPV1-6について
%   - TRPMについて
% - TRPV3について
%   - TRPV3の構造と機能
%
% 研究の目的
% - TRPV3を機能させるにいたるネットワークを明らかにする
% - TRPV3の構造と機能の関係を明らかにする
%
% 研究の方法と結果(簡単に)
%
% 研究の結論(簡単に)
%
% 論文の構成

\section{はじめに}

\section{TRPV3チャネル}
% まずTRPチャネルとTRPVチャネルについてそれぞれ1パラグラフ程度で説明した後、TRPV3チャネルについて説明する。
% TRPチャネル
TRPV3チャネルはTRP(Transient Receptor Potential)チャネル、その中でもTRPVチャネルの一種である。
TRPチャネルはカチオンチャネルのスーパーファミリーで、配列相同性はまちまちながらも似通った構造を持っており、
そのどれもが外界からの刺激を受けて応答するのに関与している。\autocite{venkatachalam_trp_2007}

% FIXME: TRPVの歴史について記述
% FIXME: TRPV5,6についても書く
TRPVチャネルファミリーはTRPV1,2,3,4,5,6に分類されており、
そのうち1-4は感知する温度や温度に対する応答が異なるものの、温度感知性を持つ。\autocite{baylie_trpv_2011}

% TRPV3チャネル概要
その中でもTRPV3チャネルは、温度T = 36-39℃で活性化する温度感知性を持つ。\autocite{baylie_trpv_2011,xuTRPV3CalciumpermeableTemperaturesensitive2002}
ヒトのゲノムデータベースを探索することで、多くのTRPV3チャネルが報告されてきた。\autocite{smithTRPV3TemperaturesensitiveVanilloid2002, xuTRPV3CalciumpermeableTemperaturesensitive2002, peier_heat-sensitive_2002}
TRPV3は人やマウスの神経細胞や肌に存在することが知られており、肌に対する外部からの物質や内部からの物資、熱によって活性化される。% TODO: 出展
% 閲覧注意: citeした3報。
% Recurrent... -> Gly573Ser
% exome... -> 3例で Gly573Ser, 1例ずつGly573Cys, Trp692Gly
また、TRPV3が肌でこれらの刺激を感受するため、肌の痛みや「かゆみ」などの感覚を与えるほか、
TRPV3遺伝子の変異はOlmsted syndromeという遺伝性皮膚疾患を引き起こす。\autocite{lin_exome_2012,lai-cheong_recurrent_2012,nilius_trpv_2013}
TRPV3は構造が明らかになる前から、活性・非活性を膜間を流れる電流によって検出することで、活性化条件を特定する研究が行われてきた。
\autocite{smithTRPV3TemperaturesensitiveVanilloid2002,xuTRPV3CalciumpermeableTemperaturesensitive2002,nadezhdinStructuralMechanismHeatinduced2021,chung_2-aminoethoxydiphenyl_2004}
% FIXME: 本当はTRPV3の性質についてもう少し詳しく書きたい。
% FIXME: 時間があればパラグラフを分けてもいい。
TRPV3は他のTRPチャネルに対して特異的な点として、温度刺激を繰り返し与えることによりイオンチャネルとしての活性が高まる。
\autocite{peier_heat-sensitive_2002,chung_2-aminoethoxydiphenyl_2004,liu_hysteresis_2011}

% 機能と構造について説明
TRPV3は単量体は790個程度のアミノ酸残基で構成された膜タンパク質であり、ホモ4量体を形成する。(図\ref{fig:trpv3})

\begin{figure}
  \centering
  \begin{subfigure}{0.4\textwidth}
    \includegraphics[width=\textwidth]{trpv3}
    \caption{膜の外側から見た様子}
    \label{fig:trpv3}
  \end{subfigure}
  \begin{subfigure}{0.4\textwidth}
    % FIXME: もっといいアングルや脂質を含めた様子がほしい。
    \includegraphics[width=\textwidth]{trpv3_angle}
    \label{fig:trpv3_angle}
  \end{subfigure}
  \caption{TRPV3の一つである7MIOの構造}
  \label{fig:trpv3_structures}
\end{figure}

残基数が多く構造が大きいためクライオ電子顕微鏡が登場するまでは一部のドメインのみの構造解析にとどまっていたが、
% https://www.rcsb.org/search?request=%7B%22query%22%3A%7B%22type%22%3A%22group%22%2C%22nodes%22%3A%5B%7B%22type%22%3A%22group%22%2C%22nodes%22%3A%5B%7B%22type%22%3A%22group%22%2C%22nodes%22%3A%5B%7B%22type%22%3A%22terminal%22%2C%22service%22%3A%22full_text%22%2C%22parameters%22%3A%7B%22value%22%3A%22TRPV3%22%7D%7D%2C%7B%22type%22%3A%22terminal%22%2C%22service%22%3A%22full_text%22%2C%22parameters%22%3A%7B%22value%22%3A%22TRPV%203%22%7D%7D%5D%2C%22logical_operator%22%3A%22and%22%7D%5D%2C%22logical_operator%22%3A%22and%22%2C%22label%22%3A%22full_text%22%7D%2C%7B%22type%22%3A%22group%22%2C%22nodes%22%3A%5B%7B%22type%22%3A%22group%22%2C%22nodes%22%3A%5B%7B%22type%22%3A%22group%22%2C%22nodes%22%3A%5B%7B%22type%22%3A%22terminal%22%2C%22service%22%3A%22text%22%2C%22parameters%22%3A%7B%22attribute%22%3A%22exptl.method%22%2C%22value%22%3A%22ELECTRON%20MICROSCOPY%22%2C%22operator%22%3A%22exact_match%22%7D%7D%5D%2C%22logical_operator%22%3A%22or%22%2C%22label%22%3A%22exptl.method%22%7D%5D%2C%22logical_operator%22%3A%22and%22%7D%5D%2C%22logical_operator%22%3A%22and%22%2C%22label%22%3A%22text%22%7D%5D%2C%22logical_operator%22%3A%22and%22%7D%2C%22return_type%22%3A%22entry%22%2C%22request_options%22%3A%7B%22paginate%22%3A%7B%22start%22%3A0%2C%22rows%22%3A25%7D%2C%22results_content_type%22%3A%5B%22experimental%22%5D%2C%22sort%22%3A%5B%7B%22sort_by%22%3A%22rcsb_accession_info.initial_release_date%22%2C%22direction%22%3A%22asc%22%7D%5D%2C%22scoring_strategy%22%3A%22combined%22%7D%2C%22request_info%22%3A%7B%22query_id%22%3A%227341f71e2e418e1c5dd1cf5281aa6898%22%7D%7D
% これ https://www.nature.com/articles/s41594-018-0108-7 が初出のはず。1,2月後にこの論文だけciteしてる悔しい論文があった。
% 最新の研究は2023年9月に発表されていたので、現在進行形でいいと思う。
2018年にクライオ電子顕微鏡法を用いて全長構造の構造解析が行われた\autocite{singhStructureGatingMechanism2018,zubcevic_conformational_2018}ことを皮切りに構造解析の研究が進んでいる。

% 機能と構造の関係について
% TODO: 絶対書く

\section{情報伝達}
% EEN等の先行研究を紹介する。

\section{計算手法}
% - 熱伝導度と熱流について述べる
本研究では、先行研究が示したEEN、\Delta EEN、r\Delta EENの概念\autocite{ishikuraEnergyExchangeNetwork2015,ota_energy_2019,poudel_energy_2022}を
熱流解析に応用して、TRPV3の活性化と調節における構造的および機能的ダイナミクスを明らかにする。
なお、本研究の目的を達成するには熱流にこだわらずEnergy fluxやエネルギー伝導度を用いた方法を用いてもよいが
熱流から計算される熱伝導度は、実験的に測定することができるというメリットがあるため、本研究では熱流を用いた手法を用いることにした。

先行研究でエネルギー流ではなくエネルギー伝導度を用いていたことにならい、本研究でも熱流ではなく熱伝導度を用いて解析した。
ここで、熱伝導度と熱流について簡単に説明する。

% TODO: 改善。
熱伝導度$\lambda$は平衡状態において。

\begin{equation}
  \label{eq:thermal_conductivity}
  \lambda = \frac{1}{3Vk_B T^2} \int_{0}^{\infty} \left\langle
    \bm{h}(0) \cdot \bm{h}(t)
  \right\rangle dt
\end{equation}

ここで$\mathbf{h}$は熱流を表す。\autocite{mcquarrie_statistical_2000}

熱流$\mathbf{h}$をこう定義する。

\begin{equation}
  \label{eq:heat_flow}
  \bm{h} \equiv \frac{d}{dt} \left\{
    \sum_{i=1}^{N} \bm{r}_i \cdot E_i
  \right\} = \sum_{i=1}^{N} \left\{
    \frac{d\bm{r_i}}{dt}E_i + \bm{r_i}\frac{dE_i}{dt}
  \right\}
\end{equation}

ここで、右辺の $\frac{dE_i}{dt}$は

\begin{equation}
  \label{time_derivative_of_energy}
  \frac{dE_i}{dt} = \sum_{j=1}^{N} \frac{1}{2} \bf{F}_{ij} \cdot (\bf{v}_i + \bf{v}_j)
\end{equation}

と展開できる\autocite{leitner_mapping_2018}、また、$\frac{d\bm{r_i}}{dt}E_{i}$が十分小さいと仮定すれば、式\ref{eq:heat_flow}は次のように変形できる。\autocite{yamatoComputationalStudyThermal2022,oai:nagoya.repo.nii.ac.jp:02007698}

\begin{align}
  \label{eq:heat_flow_expanded}
  \bm{h} &= \sum_{i=1}^{N} \left\{
              \bm{r_i}\sum_{j=1, j \neq i }^{N} \frac{1}{2} \bf{F}_{ij} \cdot (\bf{v}_i + \bf{v}_j)
            \right\} \\
         &= \sum_{i=1}^{N} \sum_{j>i}^{N}
              (\bm{r_i} - \bm{r_j}) \frac{1}{2} \bf{F}_{ij} \cdot (\bf{v}_i + \bf{v}_j)
         &= \sum_{(i, j)}
              (\bm{r_i} - \bm{r_j}) \frac{1}{2} \bf{F}_{ij} \cdot (\bf{v}_i + \bf{v}_j)
\end{align}

ここで、一行目から二行目への式変形には $\bf{F}_{ij} = - \bf{F}_{ij}$を用いた。

最後に、