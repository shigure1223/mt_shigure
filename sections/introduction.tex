\section{はじめに}

タンパク質はアミノ酸が多数繋がって構成されている高分子化合物であり、タンパク質全体が分子機械として働く。しかしこのアミノ酸単位やアミノ酸間の相互作用という”部分”としての局所的挙動とドメイン単位やタンパク質という”全体”としての大域的挙動には時空間の大きな隔たりがある。
それが顕著に表れている具体的な話でいうと、タンパク質のアロステリー現象が挙げられる。
アロステリーとは、分子の一部が離れた他の部分に影響を及ぼす現象であり、タンパク質の機能を制御する重要な特性である。具体的には、リガンド結合など外部刺激の効果によってシグナルが残基間相互作用を介して活性部位に伝わり、タンパク質を活性化させる。
この過程は、サブピコ秒からミリ秒の時間スケールにわたるダイナミクスと、サブ\text{\AA}~数十\text{\AA}の空間スケールの相互作用が連動して行われることが興味深い点であり大きな謎となっている。
%参考文献:アロステリー、時間スケールと空間スケール
%2005「シグナル伝達のアロステリック機構」
%https://www.science.org/doi/abs/10.1126/science.1108595
アロステリーのメカニズムを解明するの一つの方法として、\textbf{グラフ理論}を用いたアプローチが注目されている。
グラフ理論は、分子内の残基間の相互作用をネットワークとして表現することができ、複雑な動的挙動を解析するための協力なツールとして広く利用されてきた。
アロステリー機構に関しては、中心性の高い残基やネットワーク上における最短経路上によく現れる残基を解析することで、タンパク質のシグナル伝達の効率的な経路の同定に有用な知見を与えてきた。
%参考文献:グラフ理論
%2011「タンパク質構造の残基ネットワークの解析と可視化」
%https://www.cell.com/trends/biochemical-sciences/abstract/S0968-0004(11)00013-2?_returnURL=https%3A%2F%2Flinkinghub.elsevier.com%2Fretrieve%2Fpii%2FS0968000411000132%3Fshowall%3Dtrue
%2011「RING: タンパク質構造における相互作用残基、進化情報、エネルギーのネットワーク化」
%https://academic.oup.com/bioinformatics/article/27/14/2003/193801?login=false
%2012「生物学的ネットワークとタンパク質構造のトポロジー解析とインタラクティブな可視化」
%https://www.nature.com/articles/nprot.2012.004

%従来のグラフ理論の説明
しかし、従来のグラフ理論に基づく解析方法では、タンパク質の動的変化を十分に反映できなかったり、シグナルが特定の最短経路を通るといういわば決定論すぎるという問題点がある。

そこで本研究では、アロステリーの解析対象として\textbf{β2アドレナリン受容体(β2AR)}を選定し、inactive状態とactive状態におけるトラジェクトリー解析を基盤とした「動的ネットワーク」の構築した。さらに、部分的に確率的な性質を反映したlouvain法による「コミュニティ検出」を通じて、アロステリック効果に関連する分子内シグナル伝達の機構の解明に努めた。
その結果、active状態で新たに形成されたコミュニティが確認され、密度を用いた定量的な解析によってその役割が明らかになった。
また、\textbf{β2アドレナリン受容体(β2AR)}特有の保存された水の重要性も示唆される結果を得ることができた。


本論文の構成は以下の通りである。
はじめに、本研究の対象タンパク質であるβ2ARについて\ref{sec:b2ar}節で記述する。
次に、グラフ理論を本研究にどのように適用したかを\ref{sec:graph theory}節で説明する。
\ref{chap:methods}章以降は、シミュレーションと解析の方法とその結果、考察をまとめる。

\newpage

\section{β2アドレナリン受容体(β2AR)}\label{sec:b2ar}
β2アドレナリン受容体(β2AR)は、Gタンパク質共役受容体(GPCR)の一種であり、細胞膜に存在する重要な受容体である。β2ARは7回膜貫通構造を持ち、リガンド(アドレナリンやノルアドレナリンなど)と結合することで構造的変化を引き起こし、細胞内のシグナル伝達を調節する。
β2ARのシグナル伝達機構は、細胞外の刺激(アドレナリンやノルアドレナリン)を細胞内のシグナルに変換するプロセスである。
アドレナリンやノルアドレナリンといったリガンド結合により、β2ARは構造的な変化を起こし、Gタンパク質と結びつき、下流のシグナル伝達経路を活性化させる。この過程を通じて、細胞内の様々なシグナルが伝達され、最終的にエネルギー供給や筋肉緊張の調節、呼吸制御、循環の改善など、幅広い生理的応答が引き起こされる。これらの機能は、心血管疾患や喘息などの治療薬開発において重要なターゲットとなっており、β2ARは医薬品開発の重要な候補分子でもある。
%詳細の構造と、リガンド・gタンパク質結合部位を示す

\subsubsection*{β2ARの初期構造および特有の残基} 
β2ARの構造は、X線結晶解析やNMR解析によって明らかにされており、その構造的特徴には、アロステリックシグナル伝達に関与する特有の残基が存在する。これらの残基の相互作用を理解することが、アロステリーのメカニズム解明に繋がると考えられる。
%モチーフと水の話

\section{グラフ理論}\label{sec:graph theory}
\documentclass{article}
\usepackage{amsmath}

\begin{document}

\section*{グラフ理論の基礎}

グラフ理論は、ローカルな情報とグローバルな情報の両方を活用して、シグナル伝達やアロステリック制御に重要な役割を果たすタンパク質の残基(=機能的残基)を同定する手法として導入されている。このアプローチは、タンパク質の構造やダイナミクスをネットワークの観点から捉えることで、直感的に理解しやすい特徴を提供している。
グラフ理論においては、タンパク質内のアミノ酸残基の相互作用を、ノードとエッジを用いたアミノ酸ネットワークとして表現することができる。

\begin{itemize}
    \item \textbf{ノード}:アミノ酸残基(原子やタンパク質全体も可能)
    \item \textbf{エッジ}:ノード同士の相互作用
\end{itemize}

この方法を用いて、タンパク質内の相互作用の構造を視覚化し、機能的に重要な残基を特定することができる。特に、機能的残基の同定には以下の二つの手法が一般的である。

\subsection*{静的ネットワークと最短距離解析}

1. \textbf{静的ネットワーク}:構造データから得られた時間依存しない変数を重みとしてネットワークを構築する。このネットワークはタンパク質の静的な相互作用を反映し、構造に基づく機能的残基を同定するために用いられる。
2. \textbf{最短距離解析}:最短経路上でよく現れる残基を、機能的に重要な残基として同定する手法である。これにより、タンパク質内でシグナルを伝達する上で重要な経路を特定することができる。

しかし、これらの手法にはいくつかの限界がある。

\subsection*{問題点}

1. \textbf{タンパク質の動的性質を反映できない}  
   静的ネットワークや最短経路解析は、タンパク質の動的変化を十分に反映できない。アロステリック変化(例えば、リガンド結合による構造変化や、Gタンパク質との相互作用によるコンフォメーション変化)のメカニズムを捉えることが難しい。
2. \textbf{最短経路以外の潜在的経路の無視}  
   最短経路解析では、タンパク質内でシグナルが伝達される経路の中で、エネルギー的に有利な経路や微細な構造変化を通じて伝わる経路を反映することができんじ。このため、最短経路以外の潜在的な経路や全体構造を反映できないという問題がある。
3. \textbf{タンパク質全体としての協調的な動きの捉えにくさ}  
   アロステリーはタンパク質の複数の部分が連携して機能する現象であるため、タンパク質全体としての協調的な動きや、多様な経路による相互作用を捉えることが難しい。

これらの問題を解決するためには、より動的で柔軟な手法を用いる必要がある。