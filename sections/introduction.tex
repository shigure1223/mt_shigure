% 研究の背景、動機、目的を述べる。
% 論文全体の見通しを示す。何章で何を述べるかを明確にする。

% 研究の背景
% - 生体と温度の関係
% - TRPチャネル全体について
%   - TRPV1-6について
%   - TRPMについて
% - TRPV3について
%   - TRPV3の構造と機能
%
% 研究の目的
% - TRPV3を機能させるにいたるネットワークを明らかにする
% - TRPV3の構造と機能の関係を明らかにする
%
% 研究の方法と結果(簡単に)
%
% 研究の結論(簡単に)
%
% 論文の構成

\section{はじめに}

生物は外部からの刺激に対して、温度変化に敏感に反応し、これに対応することで生存に必要な生理機能を維持している。

TRPチャネルは、物質、温度、浸透圧の変化などを受け取り、細孔を開閉することによって細胞内外のイオンの流れを制御する膜タンパク質である。
TRPチャネルは、特に温度の変化を感受するものがあるという点で、他のイオンチャネルとは異なる。
TRPチャネルのとりわけTRPVチャネルは、温度に応答する特異性から刺激に対する応答と構造の関係性が明らかにされており、
複数のドメインが相互作用して機能することが知られている。
しかし、これら複数のドメインがどのように相互作用して機能するのかは、未だに解明されていない。

本研究では、ドメイン間の相互作用を調べるために、閉じた状態と開いた状態において熱流解析を行った。
熱流解析以前にエネルギー流を用いたEnergy Exchange Networkというモデルが提案されており、
本研究はこの考えを熱流解析にも適応した。
その結果、\openFortyTwo と\closeFortyTwo において、ドメインが熱流ネットワークにおいて大きく二つに分割されていることがわかった。


本論文の構成は以下の通りである。
はじめに、\ref{sec:trpv3}節で本研究が対象とする物質であるTRPV3チャネルについて詳しく述べた。
次に\ref{sec:information_transmission}節で本研究が参考としたモデルであるEnergy Exchange Networkについて説明した。
\ref{chap:methods}章以降は、本研究で行ったシミュレーションの方法と結果、考察をまとめる。

\newpage

\section{TRPV3チャネル}\label{sec:trpv3}
% まずTRPチャネルとTRPVチャネルについてそれぞれ1パラグラフ程度で説明した後、TRPV3チャネルについて説明する。
% TRPチャネル
TRPV3チャネルはTRP(Transient Receptor Potential)チャネル、その中でもTRPVチャネルの一種である。
TRPチャネルはカチオンチャネルのスーパーファミリーで、配列相同性はまちまちながらも似通った構造を持っており、
そのどれもが外界からの刺激を受けて応答するのに関与している。\autocite{venkatachalam_trp_2007}

% FIXME: TRPVの歴史について記述
% FIXME: TRPV5,6についても書く
TRPVチャネルファミリーはTRPV1,2,3,4,5,6に分類されており、
そのうち1-4はThermo-TRPとも呼ばれ、感知する温度や温度に対する応答が異なるものの、温度感知性を持つ。\autocite{baylie_trpv_2011}

% TRPV3チャネル概要
その中でもTRPV3チャネルは、温度T = 36-39℃で活性化する温度感知性を持つ。\autocite{baylie_trpv_2011,xuTRPV3CalciumpermeableTemperaturesensitive2002}
ヒトのゲノムデータベースを探索することで、多くのTRPV3チャネルが報告されてきた。\autocite{smithTRPV3TemperaturesensitiveVanilloid2002, xuTRPV3CalciumpermeableTemperaturesensitive2002, peier_heat-sensitive_2002}
TRPV3は人やマウスの神経細胞や肌に存在することが知られており、肌に対する外部からの物質や内部からの物資、熱によって活性化される。% TODO: 出展
% 閲覧注意: citeした3報。
% Recurrent... -> Gly573Ser
% exome... -> 3例で Gly573Ser, 1例ずつGly573Cys, Trp692Gly
また、TRPV3が肌でこれらの刺激を感受するため、肌の痛みや「かゆみ」などの感覚を与えるほか、
TRPV3遺伝子の変異はOlmsted syndromeという遺伝性皮膚疾患を引き起こす。\autocite{lin_exome_2012,lai-cheong_recurrent_2012,nilius_trpv_2013}
TRPV3は構造が明らかになる前から、活性・非活性を膜間を流れる電流によって検出することで、活性化条件を特定する研究が行われてきた。
\autocite{smithTRPV3TemperaturesensitiveVanilloid2002,xuTRPV3CalciumpermeableTemperaturesensitive2002,nadezhdinStructuralMechanismHeatinduced2021,chung_2-aminoethoxydiphenyl_2004}
% FIXME: 本当はTRPV3の性質についてもう少し詳しく書きたい。
% FIXME: 時間があればパラグラフを分けてもいい。
TRPV3は他のTRPチャネルに対して、温度刺激を繰り返し与えることによりイオンチャネルとしての活性が高まるという特性がある。
\autocite{peier_heat-sensitive_2002,chung_2-aminoethoxydiphenyl_2004,liu_hysteresis_2011}

% 機能と構造について説明
TRPV3は単量体は790個程度のアミノ酸残基で構成された膜タンパク質であり、ホモ4量体を形成する。(図\ref{fig:trpv3})

\begin{figure}
  \centering
  \begin{subfigure}{0.4\textwidth}
    \includegraphics[width=\textwidth]{trpv3}
    \caption{膜の外側から見た様子}
    \label{fig:trpv3}
  \end{subfigure}
  \begin{subfigure}{0.4\textwidth}
    % FIXME: もっといいアングルや脂質を含めた様子がほしい。
    \includegraphics[width=\textwidth]{trpv3_angle}
    \label{fig:trpv3_angle}
  \end{subfigure}
  \caption{TRPV3の一つである7MIOの構造}
  \label{fig:trpv3_structures}
\end{figure}

残基数が多く構造が大きいためクライオ電子顕微鏡が登場するまでは一部のドメインのみの構造解析にとどまっていたが、
% https://www.rcsb.org/search?request=%7B%22query%22%3A%7B%22type%22%3A%22group%22%2C%22nodes%22%3A%5B%7B%22type%22%3A%22group%22%2C%22nodes%22%3A%5B%7B%22type%22%3A%22group%22%2C%22nodes%22%3A%5B%7B%22type%22%3A%22terminal%22%2C%22service%22%3A%22full_text%22%2C%22parameters%22%3A%7B%22value%22%3A%22TRPV3%22%7D%7D%2C%7B%22type%22%3A%22terminal%22%2C%22service%22%3A%22full_text%22%2C%22parameters%22%3A%7B%22value%22%3A%22TRPV%203%22%7D%7D%5D%2C%22logical_operator%22%3A%22and%22%7D%5D%2C%22logical_operator%22%3A%22and%22%2C%22label%22%3A%22full_text%22%7D%2C%7B%22type%22%3A%22group%22%2C%22nodes%22%3A%5B%7B%22type%22%3A%22group%22%2C%22nodes%22%3A%5B%7B%22type%22%3A%22group%22%2C%22nodes%22%3A%5B%7B%22type%22%3A%22terminal%22%2C%22service%22%3A%22text%22%2C%22parameters%22%3A%7B%22attribute%22%3A%22exptl.method%22%2C%22value%22%3A%22ELECTRON%20MICROSCOPY%22%2C%22operator%22%3A%22exact_match%22%7D%7D%5D%2C%22logical_operator%22%3A%22or%22%2C%22label%22%3A%22exptl.method%22%7D%5D%2C%22logical_operator%22%3A%22and%22%7D%5D%2C%22logical_operator%22%3A%22and%22%2C%22label%22%3A%22text%22%7D%5D%2C%22logical_operator%22%3A%22and%22%7D%2C%22return_type%22%3A%22entry%22%2C%22request_options%22%3A%7B%22paginate%22%3A%7B%22start%22%3A0%2C%22rows%22%3A25%7D%2C%22results_content_type%22%3A%5B%22experimental%22%5D%2C%22sort%22%3A%5B%7B%22sort_by%22%3A%22rcsb_accession_info.initial_release_date%22%2C%22direction%22%3A%22asc%22%7D%5D%2C%22scoring_strategy%22%3A%22combined%22%7D%2C%22request_info%22%3A%7B%22query_id%22%3A%227341f71e2e418e1c5dd1cf5281aa6898%22%7D%7D
% これ https://www.nature.com/articles/s41594-018-0108-7 が初出のはず。1,2月後にこの論文だけciteしてる悔しい論文があった。
% 最新の研究は2023年9月に発表されていたので、現在進行形でいいと思う。
2018年にクライオ電子顕微鏡法を用いて全長構造の構造解析が行われた\autocite{singhStructureGatingMechanism2018,zubcevic_conformational_2018}ことを皮切りに構造解析の研究が進んでいる。

% 機能と構造の関係について
% TODO: 絶対書く

\section{情報伝達}\label{sec:information_transmission}
% EEN等の先行研究を紹介する。
タンパク質にはリガンドの結合やレセプターへの刺激を受け取ると、構造変化を起こしたり構造をそのままに機能が活性するものが知られている。
これらのタンパク質は、しばしば医学や創薬などのために研究されている。
そのため、タンパク質内の情報伝達や相互作用を定量的に理解するために、様々なモデルが提案されてきた。% TODO: どんな

本研究室では、タンパク質内の情報伝達をエネルギー交換ネットワーク(Energy Exchange Network, EEN)というモデルで表現している。
石倉らが提案したエネルギー交換ネットワークは、アミノ酸残基対の相対距離や揺らぎ、および相互作用のすべてが残基間のエネルギーの流れに反映されることに着目し、
その「流れやすさ」を指標としたアミノ酸残基のネットワークとしてタンパク質分子の状態を表現するモデルである。
タンパク質の分子機能を誘起する外部刺激によるネットワークパターンの再編成を解析することで機能発現機構を解析することができる。\autocite{ishikura_energy_2015}
石倉らはこのモデルを使うことで、PDZ3ドメインのC末端の\alpha ヘリックスの除去によるEENの変化が、除去された部分に近いところだけに限定されず、
リガンド結合部位にまで及ぶこと\autocite{ishikura_energy_2015}を示した。
また、太田らは、活性、非活性における構造変化の小さいFixLドメインに対してこの手法を用い、
リガンド分子のヘムが酸素と結合している状態、一酸化炭素と結合している状態、結合していない状態の
3つの状態におけるEENとその変化( \Delta EEN )を計算し、機能的に重要な部位を予測した。\autocite{ota_energy_2019}
