\section{はじめに}

タンパク質はアミノ酸が多数繋がって構成されている高分子化合物であり、タンパク質全体が分子機械として働く。しかしこのアミノ酸単位やアミノ酸間の相互作用という”部分”としての局所的挙動とドメイン単位やタンパク質という”全体”としての大域的挙動には時空間の大きな隔たりがある。
それが顕著に表れている具体的な話でいうと、タンパク質のアロステリー現象が挙げられる。
アロステリーとは、分子の一部が離れた他の部分に影響を及ぼす現象であり、タンパク質の機能を制御する重要な特性である。具体的には、リガンド結合など外部刺激の効果によってシグナルが残基間相互作用を介して活性部位に伝わり、タンパク質を活性化させる。
%参考文献:アロステリーの特性
%2009「アロステリーと協同性の再考」
%https://onlinelibrary.wiley.com/doi/10.1110/ps.03259908
つまりタンパク質分子を介した情報の伝達により、分子は活性部位から離れた場所でのリガンド結合、またはその他の微小環境の摂動を関しているということになる。
しかもこの過程は、サブピコ秒からミリ秒の時間スケールにわたるダイナミクスと、サブ\text{\AA}~数十\text{\AA}の空間スケールの相互作用が連動して行われることが興味深い点であり大きな謎となっている。
%参考文献:アロステリー、時間スケールと空間スケール
%2005「シグナル伝達のアロステリック機構」
%https://www.science.org/doi/abs/10.1126/science.1108595
%大きな謎
%2008「アロステリー:「生命の第二の秘密」の図解による定義」
%https://www.sciencedirect.com/science/article/pii/S0968000408001643
この物理的に離れた場所間のコミュニケーションを解明するの一つの方法として、\textbf{グラフ理論}を用いたアプローチが注目されている。
グラフ理論は、分子内の残基間の相互作用をネットワークとして表現し、複雑な動的挙動を解析するための協力なツールとして広く利用されてきた。
%参考文献:グラフ理論
%2011「タンパク質構造の残基ネットワークの解析と可視化」
%https://www.cell.com/trends/biochemical-sciences/abstract/S0968-0004(11)00013-2?_returnURL=https%3A%2F%2Flinkinghub.elsevier.com%2Fretrieve%2Fpii%2FS0968000411000132%3Fshowall%3Dtrue
%2011「RING: タンパク質構造における相互作用残基、進化情報、エネルギーのネットワーク化」
%https://academic.oup.com/bioinformatics/article/27/14/2003/193801?login=false
%2012「生物学的ネットワークとタンパク質構造のトポロジー解析とインタラクティブな可視化」
%https://www.nature.com/articles/nprot.2012.004
タンパク質を構造ネットワークとしてモデル化すると、シグナル伝達の流れを構造的に解釈することができるようになる。
そのようなモデルは、ネットワーク内の最短経路の存在の重要性を強調しており、それらが離れた部位間での効率的な情報伝達に寄与していることを示している。
%参考文献:最短経路
%2007「分子動力学シミュレーションと構造ネットワーク解析によるメチオニルtRNA合成酵素のコミュニケーション経路の研究
%https://www.pnas.org/doi/full/10.1073/pnas.0704459104
また、その前提のもとで、高い中心性を持つ残基やネットワーク上における最短経路上によく現れる残基を解析することで、機能的残基(例えば、活性部位残基)を同定してきた。
%参考文献:中心性
%2004「タンパク質構造のネットワーク解析により機能残基を特定」
%https://www.sciencedirect.com/science/article/pii/S0022283604013592
%参考文献:最短経路
%2006「ネットワーク通信における短い経路を維持するために重要な残基がタンパク質のシグナル伝達を媒介する」
%https://www.embopress.org/doi/full/10.1038/msb4100063
実際にこれらの残基はタンパク質の折りたたみにおける重要なアミノ酸や酵素ファミリーの活性部位残基であると関連付けられている。

しかし、アロステリーにおける情報伝達は単純な「最短経路モデル」だけでは説明できず、残基の大きな集合体としての協調的な動きや複数経路の存在が重要な役割を果たしており、この視点はアロステリーのより現実的で包括的な理解を提供する可能性がある。
実際に局所的な相互作用のまとまり(=モジュール)が遅いエネルギー拡散を担い触媒部位周辺の動的安定性を維持する役割を果たしていることを述べた研究や、アロステリック信号は複数の事前に存在する経路を通じて伝達されることでタンパク質の機能が維持されることを述べている研究が存在している。
%複数経路の論文
%2009「アロステリック機能調節の起源:複数の既存経路」
%https://www.sciencedirect.com/science/article/pii/S0969212609002500?via%3Dihub
%参考文献:バイオリンモデル
%2015「タンパク質キナーゼにおけるダイナミクス駆動型アロステリー」
%https://www.cell.com/trends/biochemical-sciences/fulltext/S0968-0004(15)00166-8

本研究では、アロステリーの解析対象として\textbf{$\beta$2アドレナリン受容体($\beta$2AR)}を選定し、inactive状態とactive状態におけるトラジェクトリー解析を基盤とした「残基相互作用ネットワーク」を構築した。
そして、部分的に確率的な性質を反映したlouvain法による「コミュニティ検出」を通じて、アロステリック効果に関連する分子内シグナル伝達の機構の解明に努めた。
その結果、active状態で新たに形成されたコミュニティが確認され、エッジ密度を用いた定量的な解析によってその役割が明らかになった。
また、\textbf{$\beta$2アドレナリン受容体($\beta$2AR)}特有の保存された水の重要性も示唆される結果を得ることができた。

本論文の構成は以下の通りである。
はじめに、本研究の対象タンパク質であ$\beta$2ARについて\ref{sec:b2ar}節で記述する。
次に、グラフ理論を本研究にどのように適用したかを\ref{sec:graph theory}節で説明する。
\ref{chap:methods}章以降は、シミュレーションとネットワーク構築の方法とその結果、考察をまとめる。

\newpage

\section{$\beta$2アドレナリン受容体($\beta$2AR)}\label{sec:b2ar}
$\beta$2アドレナリン受容体($\beta$2AR)は、Gタンパク質共役受容体(GPCR)の一種であり、細胞膜に存在する重要な受容体である。$\beta$2ARは7回膜貫通構造を持ち、リガンド(アドレナリンやノルアドレナリンなど)と結合することで構造的変化を引き起こし、細胞内のシグナル伝達を調節する。
$\beta$2ARのシグナル伝達機構は、細胞外の刺激(アドレナリンやノルアドレナリン)を細胞内のシグナルに変換するプロセスである。
アドレナリンやノルアドレナリンといったリガンド結合により、$\beta$2ARは構造的な変化を起こし、Gタンパク質と結びつき、下流のシグナル伝達経路を活性化させる。
この過程を通じて、細胞内の様々なシグナルが伝達され、最終的にエネルギー供給や筋肉緊張の調節、呼吸制御、循環の改善など、幅広い生理的応答が引き起こされる。
これらの機能は、心血管疾患や喘息などの治療薬開発において重要なターゲットとなっており、$\beta$2ARは医薬品開発の重要な候補分子でもある。
%詳細の構造と、リガンド・gタンパク質結合部位を示す

\subsubsection*{$\beta$2ARの初期構造および特有の残基} 
$\beta$2ARの構造は、X線結晶解析やNMR解析によって明らかにされており、その構造的特徴には、アロステリックシグナル伝達に関与する特有の残基が存在する。
これらの残基の相互作用を理解することが、アロステリーのメカニズム解明に繋がると考えられる。
%モチーフと水の話
