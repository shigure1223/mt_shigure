\section{\openFortyTwo と\closeFortyTwo のネットワークの比較}

まず、$\Delta \lambda_{\rm{AB}}$が大きいペアと小さいペアを調べるために、$\Delta \lambda_{\rm{AB}}$ の $\mu \pm 3 \sigma$ よりも小さい、または、大きいペアを抽出した。
図\ref{fig:network_groups_tc_Delta}にドメイン間の$\Delta \lambda_{\rm{AB}}$に着目した図を示す。
また、図\ref{fig:network_groups_tc_Delta_focus_inside}にドメイン内の$\Delta \lambda_{\rm{AB}}$に着目した図を示す。

\begin{figure}
  \centering
  \includegraphics[width=\textwidth]{network_groups_tc_Delta.png}
  \caption{\openFortyTwo と\closeFortyTwo の $\Delta \lambda_{\rm{AB}}$をネットワーク表示した。
            四角形は主鎖Bの残基。ドメインごとに色を分け、N末端側から順にARD、N-linker、pre-S1、S1-S4、S4-S5-linker、S5-PH-S6、TRP-helix、CTDと分けた。
            \autocite{pumroy_structural_2020}}
  \label{fig:network_groups_tc_Delta}
\end{figure}

まず、図\ref{fig:network_groups_tc_Delta}から、
ドメインをさらにグループ化した二つのセグメント、ARD、N-linker、pre-S1、CTDとS1-S4、S4-S5-linker、S5-PH-S6、TRP-helixに分割されていることがわかる。
TRPV3の立体構造は、膜より内側(ARD、N-linker、pre-S1、CTD)と膜内(S1-S4、S4-S5-linker、S5-PH-S6、TRP-helix)に分かれているため、おそらくこのようなグループ化が行われたと考えられる。
% TRP-helixが立体構造ではPre-S1と近接しているにも関わらず、$\Delta \lambda_{\rm{AB}}$で表したネットワークに表示されないことがわかる。


\begin{figure}
  \centering
  \includegraphics[width=\textwidth]{network_groups_tc_Delta_focus_inside.png}
  \caption{$\Delta \lambda_{\rm{AB}}$が大きいペアのうち、ドメイン内のネットワーク}
  \label{fig:network_groups_tc_Delta_focus_inside}
\end{figure}

\section{機能的に重要な残基}

% ほとんど検出されず。
