生物は温度変化に敏感に反応し、これに対応することで生存に必要な生理機能を維持している。
TRP チャネルは、温度変化を感知する膜タンパク質であり、その変化によりイオンチャネルが開口し、カチオンが細胞内に流入して電位が変化し、電気信号が生成される。
最新のクライオ電子顕微鏡と分子動力学シミュレーション技術を用いて、TRP チャネルの立体構造とその機能について詳細な解析が行われており、
例えば TRPV3 は温度変化を感知するタンパク質全体に散在するドメインを介して二段階の構造変化を示すことが明らかにされている。
しかし、これらのドメインがお互いにどのように相互作用してイオンチャネルの協同的な開口を調節しているのか、未だに解明されていない。

そこで本研究では、TRPV3 チャネルの閉じた構造と開いた構造における熱流解析を実施し、アミノ酸残基間の相互作用を特徴づけることでそれらを比較する。

まず、分子動力学シミュレーションを行い分子のトラジェクトリを取得し、熱流解析、熱伝導度の計算を行った。
熱流解析には、当研究室で開発した curpを使用したが、大規模なタンパク質に対しては計算時間が非常に大きくなる課題があった。
したがって、計算時間を短縮するための新しい手法を開発した。具体的には、座標データのアクセス時間を最適化し、時間的局所性を高めるようにプログラムを改良した。
その結果、計算時間を最大39\%短縮できることが確認された。
また、チャネルが閉じた状態と開いた状態の残基間熱伝導度を比較することで、熱流ネットワークの変化を調べた。
その結果、膜の内側部分と膜に埋まっている部分とで、熱流ネットワークに断絶があることが示唆された。
