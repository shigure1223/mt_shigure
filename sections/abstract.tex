タンパク質のアロステリー現象は、リガンド結合や外部刺激によって生じる構造変化が刺激受容部位から遠隔の活性部位に影響を及ぼす現象であり、そのメカニズム解明は生命現象の理解と創薬研究の中心的的題の一つである。
本研究では、$\beta_2$アドレナリン受容体($\beta_2$AR)のアロステリー機構を明らかにするため、アミノ酸残基が構成するネットワークを解析した。

アロステリーはタンパク質の機能を制御する重要な特性であり、リガンド結合など外部刺激のシグナルが残基間相互作用を介して活性部位に伝達する仕組みを提供する。
この過程の特徴を2点挙げる。\\
1.活性部位がサブÅ~数十Å離れた場所でのリガンド結合や微小環境の摂動を総じた情報を受け取る点。\\
2.ピコ秒オーダーの残基間エネルギー移動過程がミリ秒以上のアロステリック遷移を引き起こす点。

この解明の一つの方法として、グラフ理論を用いたアプローチが注目されている。
グラフ理論は、分子内の残基間の相互作用をネットワークとして表現し、シグナル伝達機構を解析するの強力なツールとして広く利用されてきた。
ネットワーク内の最短経路が、離れた部位間での効率的な情報伝達に寄与していることを示している。

しかし、残基間エネルギー移動過程とアロステリック遷移の時間スケールの違いを考慮すると、単純な「最短経路モデル」だけではアロステリーの情報伝達を十分に説明できない可能性がある。
実際に、残基集団の協調的な運動や、複数経路の存在を示す文献もあり、この視点はアロステリーのより現実的で包括的な理解を提供する可能性がある。

本研究では、まず分子動力学シミュレーションを用いて得られたトラジェクトリーから、残基間距離を反映したネットワークを構築した。
さらに、Louvain法によるコミュニティ検出を適用し、コミュニティによるシグナル伝達機構を定量的に解析した。

その結果、active状態において新たなコミュニティの生成が有意に認められ、これがシグナル伝達において重要な役割を果たすことが示唆された。
また、膜タンパク質内のエネルギー的に保存された水分子が果たす役割を確認した。
