タンパク質のアロステリー現象は、リガンド結合や外部刺激によって生じる構造変化が活性部位に影響を及ぼす現象であり、そのメカニズム解明は生命現象の理解と創薬研究において重要な課題である。
本研究では、Gタンパク質共役受容体(GPCR)の一種である$\beta$2アドレナリン受容体(β2AR)を対象とし、アロステリー機構を明らかにするための動的ネットワーク解析を実施した。

グラフ理論を基盤とする従来の解析が抱える、動的性質の反映不足や最短経路以外の経路の無視という問題を部分的に克服するため、分子動力学(MD)シミュレーションを用いて得られたトラジェクトリーから、時間依存的な残基間相互作用を反映したネットワークを構築した。
さらに、Louvain法によるコミュニティ検出を適用することで、タンパク質の協調的な動きによるシグナル伝達機構を多角的に解析した。

その結果、active状態では新たに形成されたコミュニティが観察され、これがアロステリックシグナル伝達において重要な役割を果たすことを示唆した。
また、保存された水分子が構造安定性およびシグナル伝達に果たす役割を確認した。

本研究は、タンパク質の動的性質を考慮したネットワーク解析手法が、アロステリー機構の理解に貢献する可能性を示すものである。