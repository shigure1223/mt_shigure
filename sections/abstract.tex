タンパク質のアロステリー現象は、リガンド結合や外部刺激によって生じる構造変化が活性部位に影響を及ぼす現象であり、そのメカニズム解明は生命現象の理解と創薬研究において重要な課題である。
本研究では、Gタンパク質共役受容体(GPCR)の一種である$\beta$2アドレナリン受容体(β2AR)を対象とし、アロステリー機構を明らかにするためのネットワーク解析を実施した。

アロステリーとは、分子の一部が離れた他の部分に影響を及ぼす現象であり、タンパク質の機能を制御する重要な特性である。
具体的には、リガンド結合など外部刺激の効果によってシグナルが残基間相互作用を介して活性部位に伝わり、タンパク質を活性化させる。
この過程に関して特徴的なことは2点ある。
1つ目は、活性部位がサブ\text{\AA}~数十\text{\AA}離れた場所でのリガンド結合、またはその他の微小環境の摂動の情報を受け取っているという点である。
2つ目は、ピコ秒オーダーの残基間のエネルギー移動過程が、ミリ秒以上のアロステリック遷移を引き起こしているという点である。
つまりこの過程は、サブピコ秒からミリ秒の時間スケールにわたるダイナミクスと、サブ\text{\AA}~数十\text{\AA}の空間スケールの相互作用が連動して行われることが興味深い点であり大きな謎となっている。

この離れた場所間のコミュニケーションを解明するの1つの方法として、グラフ理論を用いたアプローチが注目されている。
グラフ理論は、分子内の残基間の相互作用をネットワークとして表現し、複雑な動的挙動を解析するための協力なツールとして広く利用されてきた。
タンパク質を構造ネットワークとしてモデル化すると、シグナル伝達の流れを構造的に解釈することができるようになる。
そのようなモデルは、ネットワーク内の最短経路の存在の重要性を強調しており、それらが離れた部位間での効率的な情報伝達に寄与していることを示している。

しかし、ピコ秒オーダーの残基間のエネルギー移動過程とミリ秒以上のアロステリック遷移の時間スケールの違いを考慮すると、アロステリーにおける情報伝達は単純な「最短経路モデル」だけでは説明できない可能性がある。
実際に、ある残基の大きな集合体としての協調的な動きや、複数経路の存在が重要な役割を果たしていると示す文献もあり、この視点はアロステリーのより現実的で包括的な理解を提供する可能性がある。

本研究では、まず分子動力学(MD)シミュレーションを用いて得られたトラジェクトリーから、残基間距離を反映したネットワークを構築した。
さらに、Louvain法によるコミュニティ検出を適用することで、タンパク質の協調的な動きによるシグナル伝達機構を定量的に解析した。

その結果、active状態では新たに形成されたコミュニティが観察され、これがアロステリックシグナル伝達において重要な役割を果たすことが示唆された。
また、膜タンパク質内のエネルギー的に保存された水分子が構造安定性およびシグナル伝達に果たす役割を確認した。